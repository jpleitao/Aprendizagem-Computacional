\documentclass{article}
\usepackage[utf8]{inputenc}
\usepackage[T1]{fontenc}
\usepackage[portuguese]{babel}

\usepackage{indentfirst}
\usepackage{makeidx}
\usepackage{stackengine}
\usepackage{amssymb}
\usepackage{amsthm}
\usepackage{hyperref}
\usepackage{color}
\usepackage{graphicx}

\usepackage{pdfpages}
\usepackage{float}

\usepackage{booktabs}

\title{\bf{Aprendizagem Computacional - Trabalho Prático 4}\vspace{80mm}}
\author{\textbf{João Tiago Márcia do Nascimento Fernandes - 2011162899} \\
\textbf{Joaquim Pedro Bento Gonçalves Pratas Leitão - 2011150072}}
\makeindex

\begin{document}

\maketitle

\pagebreak

\renewcommand*\contentsname{Índice}
\tableofcontents

\pagebreak

\section{Introdução}

O presente trabalho visa a implementação de controladores difusos de um dado sistema, dos tipos \emph{Mamdani} e \emph{Sugeno}, e que representam um hipotético processo real, tendo como base do seu funcionamento regras da \emph{lógica difusa}.

A \emph{lógica difusa} extende o conceito de \emph{lógica booleana}, admitindo valores lógicos intermediários entre os valores de \emph{FALSO} e \emph{VERDADEIRO}. Assim, a lógica difusa permite-nos extender o conceito de conjunto, permitindo que um elemento passe a possuir um \emph{grau de pertença} num dado conjunto, cujo valor varia entre \emph{0} e \emph{1}.

Relativamente aos controladores referidos, estes foram desenvolvidos no ambiente \emph{Simulink}, em conjunto com a \emph{Fuzzy Logic Toolbox}, ambos pertencentes à plataforma \emph{Matlab}. Numa fase inicial do trabalho apenas foram desenvolvidos controladores de \emph{9} regras, tendo posteriormente aumentado o seu número de regras para um total de \emph{25}.

No presente documento pretendemos apresentar de forma mais detalhada os controladores implementados, discutindo alguns detalhes relativos à sua implementação e apresentando uma reflexão crítica sobre o desempenho e performance dos diferentes controladores implementados (\emph{Mamdani} e \emph{Sugeno}, de \emph{9} e \emph{25} regras).


\pagebreak

\section{Aplicação Desenvolvida}

%Falar do desenvolvimento da aplicação -- À partida será semelhante à do primeiro trabalho: CLI, a solicitar qual o controlador a utilizar, e depois permitir ao utilizador proceder à execução do controlador

%Falar nas principais etapas de desenvolvimento da aplicação, nomeadamente:
%	1- falar na referência variável
%	2- falar no ajuste dos factores de escala
%	3- falar nas perturbações nos actuadores

%Falar nos principais ficheiros que constituem a aplicação -- Meter isso numa secção chamada "Implementação no Simulink"

%Incluir secção que explique como executar o código desenvolvido

\pagebreak

\section{Testes da Aplicação}

%Falar nos testes que realizámos e nos resultados que obtivémos (Fazer testes com e sem perturbador nos actuadores)

%Façam o estudo de desempenho do controlador em relação aos 3 objetivos:
%	1- seguir a referência
%	2- compensar a perturbação na carga
%	3- compensar a perturbação no atuador.

\pagebreak

\section{Conclusões}

%Reflexão sobre o trabalho e sobre os resultados obtidos - À partida será de esperar que os controladores com 25 regras apresentem melhores resultados que os controladores com apenas 9 regras --> Confirmar se isso se verificou e identificar qual o controlador (Mamdani ou Sugeno) que apresentou melhores resultados

\pagebreak

\section{Anexos}

\end{document}