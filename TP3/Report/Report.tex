\documentclass{article}
\usepackage[utf8]{inputenc}
\usepackage[T1]{fontenc}
\usepackage[portuguese]{babel}

\usepackage{indentfirst}
\usepackage{makeidx}
\usepackage{stackengine}
\usepackage{amssymb}
\usepackage{amsthm}
\usepackage{hyperref}
\usepackage{color}
\usepackage{graphicx}

\usepackage{booktabs}

\title{\bf{Aprendizagem Computacional - Trabalho Prático 3}\vspace{80mm}}
\author{\textbf{João Tiago Márcia do Nascimento Fernandes - 2011162899} \\
\textbf{Joaquim Pedro Bento Gonçalves Pratas Leitão - 2011150072}}
\makeindex

\begin{document}

\maketitle

\pagebreak

\renewcommand*\contentsname{Índice}
\tableofcontents

\pagebreak

\section{Introdução}

O presente trabalho foca-se na previsão e identificação de crises epiléticas.

Pretende-se que esta previsão e identificação de crises seja feita por intermédio de uma aplicação em \emph{Matlab}, por nós desenvolvida, fazendo uso de redes neuronais na sua arquitetura interna, disponíveis na \emph{Neural Networks Toolbox} do próprio \emph{Matlab}.

Para tornar a interação do utilizador com a aplicação mais fácil e intuitiva, foi incorporada uma interface gráfica na aplicação, que permite ao utilizador escolher a rede a utilizar em cada momento, bem como as suas diferentes características. É também a partir desta interface que o utilizador pode proceder ao treino das redes criadas e à realização de testes, e consulta dos respetivos resultados.

No presente documento


\pagebreak

\section{Aplicação Desenvolvida}

Falar na aplicação desenvolvida, nas suas duas principais fazes, nos principais ficheiros e em como correr a aplicação



\subsection{Implementação em Matlab}

Falar na estrutura dos ficheiros e no ficheiro que é para correr

\subsection{Execução}



\pagebreak

\section{Treino e Testes da Aplicação}



\pagebreak

\section{Conclusões}


\pagebreak

\section{Anexos}



\end{document}