\documentclass{article}
\usepackage[utf8]{inputenc}
\usepackage[T1]{fontenc}
\usepackage[portuguese]{babel}

\usepackage{indentfirst}
\usepackage{makeidx}
\usepackage{stackengine}
\usepackage{amssymb}
\usepackage{amsthm}
\usepackage{hyperref}
\usepackage{color}


\title{\bf{Aprendizagem Computacional - Trabalho Prático 2}\vspace{80mm}}
\author{\textbf{João Tiago Márcia do Nascimento Fernandes - 2011162899} \\
\textbf{Joaquim Pedro Bento Gonçalves Pratas Leitão - 2011150072}}
\makeindex

\begin{document}

\maketitle

\pagebreak

\renewcommand*\contentsname{Índice}
\tableofcontents

\pagebreak

\section{Introdução}

Este trabalho foca-se no reconhecimento de caracteres da numeração árabe, ou seja, os caracteres 0 a 9.

Pretende-se que este reconhecimento seja realizado por uma aplicação desenvolvida em \emph{Matlab}, que faz uso de redes neuronais na sua arquitetura interna, bem como da \emph{Nerual Networks Toolbox} do próprio \emph{Matab}.

\pagebreak

\textbf{FIXME: Testar classificação de dígitos perfeitos e de dígitos não perfeitos (alguns não perfeitos são corretamente classificados, e todos os perfeitos são corretamente classificados)}

\textbf{FIXME:Responder às perguntas deles nas conclusões??}

\textbf{FIXME: Dizer que memória associativa só funciona se preenchermos a tabela toda pela ordem: 1,2,3,4,5,6,7,8,9,0 linha-a-linha}

\pagebreak
\section{}

\end{document}