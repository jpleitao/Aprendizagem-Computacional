\documentclass{article}
\usepackage[utf8]{inputenc}
\usepackage[T1]{fontenc}
\usepackage[portuguese]{babel}

\usepackage{indentfirst}
\usepackage{makeidx}
\usepackage{stackengine}
\usepackage{amssymb}
\usepackage{amsthm}
\usepackage{hyperref}
\usepackage{color}
\usepackage{graphicx}


\title{\bf{Aprendizagem Computacional - Trabalho Prático 2}\vspace{80mm}}
\author{\textbf{João Tiago Márcia do Nascimento Fernandes - 2011162899} \\
\textbf{Joaquim Pedro Bento Gonçalves Pratas Leitão - 2011150072}}
\makeindex

\begin{document}

\maketitle

\pagebreak

\renewcommand*\contentsname{Índice}
\tableofcontents

\pagebreak

\section{Introdução}

Este trabalho foca-se no reconhecimento de caracteres da numeração árabe, ou seja, os caracteres 0 a 9.

Pretende-se que este reconhecimento seja realizado por uma aplicação desenvolvida em \emph{Matlab}, que faz uso de redes neuronais na sua arquitetura interna, disponíveis na \emph{Nerual Networks Toolbox} do próprio \emph{Matlab}.

A aplicação desenvolvida visa o estudo de duas arquiteturas distintas no reconhecimento dos caracteres:

\begin{itemize}
\item Na primeira arquitetura a aplicação será constituída por uma \emph{memória associativa} e um \emph{classificador}

\item Na segunda arquitetura a aplicação apenas recorre ao \emph{classificador}
\end{itemize}

\vspace{.3cm}

As duas arquiteturas apresentadas estão presentes nas figuras que se seguem:

\begin{figure}[h]
  \centering
      \includegraphics[scale=0.4]{AM_Classifier.png}
  \caption{Arquitetura da aplicação com \emph{memória associativa + classificador}}
\end{figure}

\begin{figure}[h]
  \centering
      \includegraphics[scale=0.4]{Classifier.png}
  \caption{Arquitetura da aplicação apenas com o \emph{classificador}}
\end{figure}

Através da análise destas figuras podemos determinar um comportamento padrão para a aplicação:

\begin{itemize}
\item Numa fase inicial, os caracteres a identificar poderão, ou não, ser fornecidos à \emph{memória associativa}, que está encarregue da sua "filtragem" ou "correção": Se os caracteres fornecidos não forem perfeitos, a memória associativa aproxima-os dos respetivos caracteres perfeitos.

\item De seguida os dados, corrigidos ou não, serão fornecidos ao \emph{classificador}, que se encarregará de proceder à identificação dos mesmos.
\end{itemize}

No presente documento iremos proceder à apresentação em maior detalhe destas duas arquiteturas e das suas implementações, bem como da aplicação \emph{Matlab} desenvolvida, e de como poderá ser utilizada. Pretendemos também fazer uma análise crítica da performance da aplicação, nomeadamente da sua capacidade de classificar corretamente novos caracteres fornecidos.

\pagebreak

\section{Aplicação Desenvolvida}

\pagebreak

\subsection{Memória Associativa + Classificador}

\pagebreak

\subsection{Classificador}

\pagebreak

\subsection{Implementação em Matlab}

Indicar ficheiros criados e alterações ao código fonte (só dizer o que é que comentamos no mpaper.m)

\subsubsection{associativeMemory.m}


\subsubsection{createNetwork.m}


\subsubsection{myclassify.m}


\subsubsection{saveNetwork.m}

Este ficheiro não estamos a usar neste momento. Mantê-lo na aplicação e no relatório??

\subsection{Execução}

Explicar como executar a aplicação. Não esquecer que quando usamos a memória associativa assumimos que o utilizador desenha os caracteres 1;2;3;4;5;6;7;8;9;0 por esta ordem, em cada linha, preenchendo todas as linhas.

\pagebreak

\section{Testes e Resultados}

Descrição de como fizémos os casos de teste, dimensões, etc

\pagebreak

\section{Conclusões}

Conclusões, lolol

\pagebreak

\textbf{FIXME: Testar classificação de dígitos perfeitos e de dígitos não perfeitos (alguns não perfeitos são corretamente classificados, e todos os perfeitos são corretamente classificados)}

\textbf{FIXME: Dizer que memória associativa só funciona se preenchermos a tabela toda pela ordem: 1,2,3,4,5,6,7,8,9,0 linha-a-linha}

\vspace{.3cm}

\textbf{FIXME: Perguntas relatório:}

\begin{itemize}
\item How does the data set influence the performance of the classification system?

\item Which architecture provides better results: only the classifier or the associative memory+classifier?

\item Which is the best activation function: hardlim, linear or logsig?

\item Does the Hebb rule perform well?

\item Is the classification system able to achieve the main objectives (classification of digits)?

\item Which is the percentage of well classified digits?

\item How is the generalization capacity?

\item Is the classification system robust enough to give correct outputs when new inputs are not perfect?

\item Which is the percentage of well classified new inputs?
\end{itemize}

\pagebreak

\end{document}